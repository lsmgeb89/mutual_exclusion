\documentclass[a4paper]{report}

% load packages
%\usepackage{showframe}
\usepackage[utf8]{inputenc}
\usepackage{enumitem}
\usepackage{listings}
\usepackage{courier}
\usepackage{hyperref}

\setlength{\parskip}{1em}

\lstset{
  breaklines=true,
  basicstyle=\ttfamily,
  showstringspaces=false
}

\begin{document}

\title{Report of Programming Assignment 1 for CS 6301.001: Special Topics in Computer Science --- Introduction to Multi-Core Programming}

\author{Siming Liu}

\maketitle{}

\section*{Experiment Objective}
The experiment is intended to compare performance of three different kind of locks. There are Test-And-Set (aka TAS), Test-Test-And-Set (aka TTAS) and Tournament lock.

\section*{Experiment Method}
For each lock ($l$), I launch multiple threads ($n$) to increase a shared variable concurrently.
In each thread, it loops multiple times ($c$).
These threads use one single lock to enter and leave critical section. In the thread running function, there are just simply calling \lstinline{Lock()}, adding shared variable by one and calling \lstinline{Unlock()} and loop it by $c$ times.
A high accurate timer is used to calculate the total time from starting of these threads to end of these threads.
I call the procedure above a test unit for $l$ lock, $n$ threads and $c$ loop times, denoted by $tu(l, n, c)$.
The result of a test unit is total running time of these threads in nanoseconds.
In order to avoid outliers, I do a test unit for several times $r$ and then get an average time, denoted by $t = atu(l, n, c, r)$.

In order to fully evaluate performance, I change the number of threads from $1$ to the number of logical cores ($m$) in the machine and then get several pair values, denoted by $(n_1, t_1), (n_2, t_2), \ldots, (n_m, t_m), t_i = atu(l, n_i, c, r)$. Thus I get a line for lock $l$ by putting these pair values in the coordinate ($x$ axis: $n$, $y$ axis: $t$ nanoseconds) and linking these points together.

There are totally three locks.
For each lock, I test and get a line and combine them in a plot.
So from the plot, we could compare performance of three locks clearly.

\section*{How to Implement}
I use \lstinline{C++} and several \lstinline{C++11} features to implement it.
The program called \lstinline{lock_compare} could receive three parameters: $n$, $c$ and $r$.

Basically, for each lock, I implement a class and expose two public functions \lstinline{Lock} and \lstinline{Unlock}.
For example, \lstinline{class TASLock} for TAS, \lstinline{class TTASLock} for TTAS.
Because Tournament lock uses a tree structure in which each node is a Peterson lock, there are two classes for implementing it: \lstinline{class PetersonLock} and \lstinline{class TournamentLock}.
For Tournament class, the current level is $4$.
It is hard-coded in the source code by \lstinline{static constexpr std::size_t level_ = 4;}.
In order to achieve mutual exclusion, I use \lstinline{C++11}'s \href{http://en.cppreference.com/w/cpp/atomic}{Atomic Operations Library} in the internal variables in each lock class.

In terms of testing, I implement a base class called \lstinline{class BaseTester} to fill common logic for all tester.
And nearly for each lock, there is a tester class that is inherited from \lstinline{class BaseTester}.
In the end, there is a \lstinline{class Tester} as a only one entry for testing.
It does all necessary tests by calling each tester and output result points.
I also write a \lstinline{R} script to generate a plot by using the output of testing program.

\section*{How to Verify Mutual Exclusion}
It is simple.
After each unit test $tu(l, n, c)$, the shared variable should be equal to $n \cdot c$.
So checking this value after each unit test, if the condition is violated, I throw an exception and catch it in the top level to stop testing procedure because mutual exclusion is violated.

\section*{Testing Environment and Experiment Results}

\end{document}
